Interpreting 2D images and understanding them by creating a mental model on how would they look in the real world is a tedious and time taking process. In case of the martian planet NASA has immense number of images to understand it. The usage of 3D data (DEMs and digital models of rocks) will prove to be an advantage when viewed in 3D instead of 2D images. The projection might change the impression of the object since it will be possible to interpret the size and distance from a different perspective. The combination of 3D data with virtual reality can help in scientific explorations in geosciences and astrobiology as it is the next best thing to being on the martian planet. Thus we created a pipeline that provides flexible and dynamic tools for representing the planetary surfaces of Mars. The virtual scenes created by us are focused to be accurate and scientific. 
